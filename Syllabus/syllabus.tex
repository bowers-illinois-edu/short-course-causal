\documentclass[10pt]{article}
\usepackage{bibentry,natbib,url,comment,amsmath}
\usepackage{graphicx,parskip}
\usepackage{Styles/sometexdefs}

%\usepackage[spanish]{babel}
\usepackage[T1]{fontenc}
\usepackage{textcomp}
\usepackage[utf8]{inputenc}
\usepackage{tgtermes}
%\usepackage{tgpagella}
%\usepackage[sc]{mathpazo}

\renewcommand*{\bfdefault}{bx}
\renewcommand{\oldstylenums}[1]{%
  {\fontfamily{pplj}\selectfont #1}}
\usepackage{microtype}

\hyphenation{Incomplete}

\bibliographystyle{apalike}
\usepackage[letterpaper,bottom=.75in,top=1in,right=1in,lmargin=1.15in]{geometry}

\usepackage{advdate}

%\titleformat{〈command〉}[〈shape〉]{〈format〉}{〈label〉}{〈sep〉}{〈before-code〉}[〈after-code〉
\usepackage[compact,nobottomtitles*]{titlesec} %nobottomtitles
\titleformat{\part}[hang]{\Large\scshape}{\thepart}{5em}{}{}
\titleformat{\section}[hang]{\large\bfseries}{\thepart}{5em}{}{}
\titleformat{\subsection}[leftmargin]{\small\bfseries\filleft}{\thesubsection}{.5em}{\hspace{-.75in}}{}
\titleformat{\subsubsection}[leftmargin]{\itshape\filleft}{\thesubsubsection}{.2em}{\hspace{-.75in}}{}
\titleformat{\paragraph}[runin]{\bfseries}{\theparagraph}{0em}{}{}

\titlespacing{\part}{0ex}{.5ex plus .1ex minus .2ex}{-.25\parskip}
\titlespacing{\section}{0ex}{1.5ex plus .1ex minus .2ex}{1ex}
\titlespacing{\subsection}{0ex}{.5ex plus .1ex minus .1ex}{1ex}
\titlespacing{\subsubsection}{0ex}{.5ex plus .1ex minus .1ex}{1ex}
\titlespacing{\paragraph}{0em}{1ex}{.5ex plus .1ex minus .1ex}

%%\newenvironment{introstuff} {\setcounter{secnumdepth}{0}%
%%  \titlespacing*{\section}{-.75in}{1em}{0em}{}%
%%  \titleformat{\subsection}[leftmargin]{\itshape\filleft}{\thesubsection}{.5em}{\hspace{-.75in}}{}%
%%  \titleformat{\subsubsection}[leftmargin]{\itshape\filleft}{\thesubsubsection}{.2em}{\hspace{-.75in}}{}%
%%  \titleformat{\paragraph}[hang]{\bfseries}{\theparagraph}{0em}{}{}%
%%  \titlespacing{\subsection}{2pc}{1.5ex plus .1ex minus .2ex}{1pc}%
%%  \titlespacing{\paragraph}{0em}{1ex}{0ex plus .1ex minus .1ex}%
%%  \titlespacing{\subsubsection}{2pc}{1.5ex plus .1ex minus .2ex}{1pc}%
%%} {\setcounter{secnumdepth}{1}%
%%  \titleformat{\part}[hang]{\large\bfseries}{\hspace{-.75in}\thepart}{.5em}{}{}%
%%  \titleformat{\section}[hang]{\large\bfseries}{\hspace{-.75in}\thesection}{.5em}{}{}%
%%  \titleformat{\subsection}[leftmargin]{\small\bfseries\filleft}{\thesubsection}{.5em}{\hspace{-.75in}}{}%
%%  \titleformat{\subsubsection}[leftmargin]{\itshape\filleft}{\thesubsubsection}{.2em}{\hspace{-.75in}}{}%
%%  \titleformat{\paragraph}[runin]{\bfseries}{\theparagraph}{0em}{}{}%
%%  % \titleformat{\subsection}[hang]{\itshape}{\thesubsection}{.5em}{}{}%
%%  \titlespacing{\section}{2pc}{1.5ex plus .1ex minus .2ex}{1pc}%
%%  \titlespacing{\paragraph}{0em}{1ex}{0ex plus .1ex minus .1ex}%
%%  \titlespacing{\subsubsection}{2pc}{1.5ex plus .1ex minus .2ex}{1pc}%
%%}

% Create new title appearance
\makeatletter
\def\maketitle{%
  %\null
  \thispagestyle{empty}%
  \begin{center}\leavevmode
    \normalfont
    {\large \bfseries\@title\par}%
    {\large \@author\par}%
    {\large \@date\par}%
  \end{center}%
  \null }
\makeatother

\usepackage{fancyhdr}
% \renewcommand{\sectionmark}[1]{\markright{#1}{}}

\fancypagestyle{myfancy}{%
  \fancyhf{}
  % \fancyhead[R]{\small{Page~\thepage}}
  \fancyhead[R]{\small{Causal Inference --- IPM 2019 -- Page \thepage}}
  \fancyfoot[R]{\footnotesize{Version~of~\input{|"date"}}}
  % \fancyfoot[R]{\small{\today -- Jake Bowers}}
  \renewcommand{\headrulewidth}{0pt}
  \renewcommand{\footrulewidth}{0pt}}

\pagestyle{myfancy}

\newcommand{\entrylabel}[1]{\mbox{\textsf{#1:}}\hfil}

%% These next lines tell latex that it is ok to have a single graphic
%% taking up most of a page, and they also decrease the space arou
%% figures and tables.
\renewcommand\floatpagefraction{.9} \renewcommand\topfraction{.9}
\renewcommand\bottomfraction{.9} \renewcommand\textfraction{.1}
\setcounter{totalnumber}{50} \setcounter{topnumber}{50}
\setcounter{bottomnumber}{50} \setlength{\intextsep}{2ex}
\setlength{\floatsep}{2ex} \setlength{\textfloatsep}{2ex}

\specialcomment{com} {\begingroup\sffamily\small\bfseries}{\endgroup}
\excludecomment{com}

\title{Causal Inference: Beyond the Basics}

\author{Jake Bowers \\
  \small{jwbowers@illinois.edu \\
    Online:
    \url{http://jakebowers.org/}}
}

\date{\today}

%\usepackage[pdftex,colorlinks=TRUE,citecolor=blue]{hyperref}
\usepackage[colorlinks=TRUE,citecolor=blue]{hyperref}

\renewcommand{\bibname}{ }
% \renewcommand{\refname}{\normalsize{Required:}}
\renewcommand{\refname}{\vspace{-2em}}

\def\themonth{\ifcase\month\or
  January\or February\or March\or April\or May\or June\or
  July\or August\or September\or October\or November\or December\fi}


%
%	\DayOfWeek	expands to the day of the week ("Sunday", etc.)
%	\PhaseOfMoon	expands to the phase of the moon
%
%	Written by Martin Minow of DEC (minow%bolt.dec@decwrl.dec.com).
%
\def\DayOfWeek{%
  %
  % 	Calculate day of the week, return "Sunday", etc.
  %
  \newcount\dow				% Gets day of the week
  \newcount\leap			% Leap year fingaler
  \newcount\x				% Temp register
  \newcount\y 				% Another temp register
  %		leap = year + (month - 14)/12;
  \leap=\month \advance\leap by -14 \divide\leap by 12
  \advance\leap by \year
  %		dow = (13 * (month + 10 - (month + 10)/13*12) - 1)/5
  \dow=\month \advance\dow by 10
  \y=\dow \divide\y by 13 \multiply\y by 12
  \advance\dow by -\y \multiply\dow by 13 \advance\dow by -1 \divide\dow by 5
  %		dow += day + 77 + 5 * (leap % 100)/4
  \advance\dow by \day \advance\dow by 77
  \x=\leap \y=\x \divide\y by 100 \multiply\y by 100 \advance\x by -\y
  \multiply\x by 5 \divide\x by 4 \advance\dow by \x
  %		dow += leap / 400
  \x=\leap \divide\x by 400 \advance\dow by \x
  %		dow -= leap / 100 * 2;
  %		dow = (dow % 7)
  \x=\leap \divide\x by 100 \multiply\x by 2 \advance\dow by -\x
  \x=\dow \divide\x by 7 \multiply\x by 7 \advance\dow by -\x
  \ifcase\dow Sunday\or Monday\or Tuesday\or Wednesday\or
  Thursday\or Friday\or Saturday\fi
}

\usepackage{enumitem}% http://ctan.org/pkg/enumitem


\begin{document}
\maketitle

%\begin{introstuff}


  \part*{Overview}

  \subsection*{Where/When}

  We will meet Tuesday and Wednesday, July 16 and 17, 0900--1200.

  \subsection*{Introduction}

  This course builds on the Introduction to Causal Inference course given
  earlier in the Institute of Political Methodology Summer School of 2019. We
  will focus on deepening and applying the concepts introduced in that course,
  especially with regards to the use and diagnosis of matching techniques for non-parametric
  adjustment in non-randomized research designs and design-based statistical
  inference based on the resulting matched research designs.

  \part*{Schedule}

%\end{introstuff}

\section*{Class 1: Doing Matching, Assessing Matched Designs, Estimating Effects}

\subsection*{Topics:}

\paragraph*{How to use optimal, full matching to produce a matched-research
design?} \vspace{-1.5em} Multivariate optimal matching review using \texttt{optmatch}  in R:
producing matched research designs using matching on scalars, propensity
scores, and Mahalanobis distances.

\paragraph*{How to reason about whether we have a good matched research design?}
Multivariate balance assessment using null hypothesis testing using the
\texttt{RItools} package for R (and perhaps also using equivalence testing).

\paragraph*{How to customize and focus the matched research design creation?}
Multivariate optimal matching; calipers; penalties; combining scores


\paragraph*{How to estimate the ATE  and test hypotheses about causal effects
given a matched research design?} Estimating ATE and Std Errors from Matched
Designs using the Block Randomized Experiment as an Analogy.

\subsection*{References}

\citealp[Chap 1,3,7,8,9,13]{rosenbaum2010design}  (\url{http://www.springerlink.com/content/978-1-4419-1212-1/contents/})

\citealp[Chap 9.0--9.2]{gelman2007dau} (on causal inference and the problems of interpolation and extrapolation)

\citealp{hansen:2004} on full matching for adjustment

\citealp{hansen2008cbs} on assessing balance.


\section*{Class 2: Going Beyond Two Treatment Groups}

\subsection*{Topics:}

\paragraph*{Matching with more than one treatment}  \vspace{-1.5em} Multivariate optimal \textbf{nonbipartite} matching review using the \texttt{nbpMatching} package for R: producing matched research designs using matching on scalars, propensity scores, and Mahalanobis distances.
\paragraph*{Balance assessment after non-bipartite matching.} Multivariate balance assessment using null hypothesis testing (and using equivalence testing).
\paragraph*{Focusing and customizing non-bipartite matched designs} Multivariate optimal matching; calipers; penalties; combining scores
\paragraph*{Statistical inference after non-bipartite matching} Estimating ATE and Std Errors from Matched Designs using the Block Randomized Experiment as an Analogy.

\subsection*{References:}

\citealp[Chap 11 \& 12]{rosenbaum2010design}

\citealp{lu2011optimal}

\section*{References}

\bibliography{references}

  \end{document}

